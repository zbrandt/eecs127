\section{Set Theory}

\subsection{Basic Set Properties}
A set $\mathcal{S} \subseteq \R^n$ is \textbf{open} if for
every $x \in \mathcal{S}$, $\exists \epsilon > 0$ s.t.
$B_\epsilon(x) \subset \mathcal{S}$, where
$B_\epsilon(x)$ is a ball centered at $x$ with radius
$\epsilon$.

$\mathcal{S} \subseteq \R^n$ is \textbf{closed} if its
complement $\R^n \setminus \mathcal{S}$ is open.

$\mathcal{S} \subseteq \R^n$ is \textbf{bounded} if
$\exists r > 0$ s.t. $\mathcal{S} \subseteq B_r(0)$.

A set is \textbf{compact} if it is closed and bounded.

\textbf{Interior} of $\mathcal{S}$: points
$x \in \mathcal{S}$ s.t. we can draw a ball in $\R^n$
centered at $x$ of non-zero radius that belongs to
$\mathcal{S}$. Denoted as $\text{int}\,\mathcal{S}$.

\textbf{Closure}: $\text{cls}(\mathcal{S}) = \{z \in \R^n
\mid z = \lim_{k \to \infty} x^{(k)} \text{ where }
x^{(k)} \in \mathcal{S}, \forall k\}$.

\textbf{Boundary}:
$\partial \mathcal{S} = \text{cls}(\mathcal{S}) \setminus
\text{int}(\mathcal{S})$.

\subsection{Affine \& Convex Sets}
\textbf{Affine combination} of
$x_1,\ldots,x_k \in \R^n$:
$\{\sum_{i=1}^k \alpha_i x_i \mid \sum_{i=1}^k \alpha_i = 1\}$.

\textbf{Convex combination}: $\{\sum_{i=1}^k \alpha_i x_i
\mid \sum_{i=1}^k \alpha_i = 1, \alpha_i \geq 0, \forall i\}$.

$\mathcal{S}$ is \textbf{affine} if for all
$x,y \in \mathcal{S}$ and $t \in \R$, the affine
combination $tx + (1-t)y$ is in $\mathcal{S}$ (affine sets
are based on subspaces). A hyperplane is an affine set, but
a half-space is not.

$\mathcal{S}$ is \textbf{convex} if for all
$x,y \in \mathcal{S}$ and $t \in [0,1]$, the convex
combination $tx + (1-t)y$ is in $\mathcal{S}$.

A polyhedron $\{x \mid a_i^\top x \leq b_i, c_j x = d,
\forall i,j\}$ is convex. Norm balls and half-spaces are
convex. The set of PD matrices is convex, and the set of
PSD matrices is also convex.

\textbf{Affine hull} of a set: smallest affine set
containing the set. It is the set of affine combinations of
any $k$ points in the set.

\textbf{Convex hull} of a set: smallest convex set
containing the set. It is the set of convex combinations of
any $k$ points in the set.

\textbf{Operations preserving convexity}: (1) Intersection
of convex sets is convex (note union may not be convex). (2)
Affine transformation: $\mathcal{S} = \{f(x): x \in
\mathcal{S}\}$ is convex if $f: \R^n \to \R^m$ is affine
and $\mathcal{S}$ is convex. (3) Projections of convex sets
are convex.

\subsection{Dimension \& Relative Interior}
\textbf{Dimension} of a set $\mathcal{S} \subseteq \R^n$:
If $\mathcal{S}$ is a subspace, dimension is minimum number
of spanning vectors. If $\mathcal{S}$ is affine,
$\mathcal{S} = x_0 + V$ where $V$ is a subspace, and the
dimension is the dimension of $V$. If $\mathcal{S}$ is
convex, dimension is defined as the dimension of the affine
hull of $\mathcal{S}$.

\textbf{Relative interior} of convex set
$\mathcal{S} \subseteq \R^n$: a point
$x \in \mathcal{S}$ is in the relative interior if we can
draw a ball in the affine hull of $\mathcal{S}$ centered at
$x$ of non-zero radius that belongs to $\mathcal{S}$.
Denoted as $\relint \mathcal{S}$.

\subsection{Separating Hyperplane}
\textbf{Hyperplane}: $(n-1)$ dimensional affine set, can be
written as $H = \{z \in \R^n \mid a^\top z = b\}$ for a
non-zero vector $a \in \R^n$ and scalar $b$.

$a$ is the \textbf{normal vector} of the hyperplane. For any
two vectors $z^1, z^2 \in H$, we have
$a \perp (z^1 - z^2)$.

Hyperplanes divide $\R^n$ into half-spaces:
$H_- = \{x \mid a^\top x \leq b\}$ and
$H_+ = \{x \mid a^\top x \geq b\}$.

\textbf{Supporting hyperplane theorem}: For a convex set
$C$ and boundary point $z \in \partial C$, we can always
find a supporting hyperplane
$H = \{x \in \R^n \mid a^\top x = b\}$ satisfying: (1)
$z \in H$, (2) $C \subseteq H_-$, where
$H_- = \{x \in \R^n \mid a^\top x \leq b\}$.

\textbf{Separating hyperplane}: A hyperplane
$H = \{x \in \R^n \mid a^\top x = b\}$ separates
$C_1$ and $C_2$ if (1) $C_1 \subseteq H_-$, where
$H_- = \{x \mid a^\top x \leq b\}$, (2)
$C_2 \subseteq H_+$, where $H_+ = \{x \mid a^\top x \geq b\}$.

If $H \cap C_1 = H \cap C_2 = \emptyset$, then $H$ strictly
separates $C_1$ and $C_2$.

\textbf{Separating hyperplane theorem}: Assume $C_1, C_2$
are convex. Two statements: (1) If
$C_1 \cap C_2 = \emptyset$, then a separating hyperplane
exists. (2) If $C_1 \cap C_2 = \emptyset$, $C_1$ and $C_2$
are closed, and either $C_1$ or $C_2$ are bounded, then a
strictly separating hyperplane exists.
