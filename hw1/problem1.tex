\begin{homeworkProblem}
    The Fruit Computer company produces two types of computers: Pear computers 
    and Apricot computers. The following table shows the number of hours and 
    the number of chips needed to make a computer as well as the equipment cost
    and selling price:
    \\
    
    \begin{tabular}{ l l l l l }
        \hline
        Computer & Labor & Chips & Equipment cost per unit (\$) & Selling price (\$) \\ \hline
        Pear & 1 hour & 3 & 55 & 400 \\ 
        Apricot & 2 hours & 6 & 100 & 900 \\
        \hline
    \end{tabular}
    \\ \\
    
    A total of 3000 chips and 1300 hours of labor are available. The company 
    needs to decide how many computers from each type should be made in order to 
    maximize the profit.
    \\ \\
    Formulate this problem as an optimization problem.
    \\ \\
    \begin{solution}
        \[
        \begin{array}{rl}
        \min\limits_{\boldsymbol{x}} & \underset{f_0(\boldsymbol{x})}{\boxed{-400x_p - 900x_a + 55x_p + 100x_a}}, \\ [3ex]
        \text{s.t.} & \underset{f_1(\boldsymbol{x})}{\boxed{x_p + 2x_a - 1300}} \leq 0, \\ [3ex]
                    & \underset{f_2(\boldsymbol{x})}{\boxed{3x_p + 6x_a - 3000}} \leq 0, \\ [3ex]
                    & \underset{f_3(\boldsymbol{x})}{\boxed{-x_p}} \leq 0, \\ [3ex]
                    & \underset{f_4(\boldsymbol{x})}{\boxed{-x_a}} \leq 0
        \end{array}
        \]
        where
        \begin{itemize}
            \item $(x_p, x_a) \in \mathbb{Z}^2$ is the \textit{decision variable};
            \item $f_0 : \mathbb{R}^2 \to \mathbb{R}$ is the \textit{objective function}, or \textit{cost};
            \item $f_1 : \mathbb{R}^2 \to \mathbb{R}$ represents the \textit{labor-hour constraint};
            \item $f_2 : \mathbb{R}^2 \to \mathbb{R}$ represents the \textit{chip constraint};
            \item $f_3$ \& $f_4$ : $x_p$ and $x_a$ must be non-negative;
        \end{itemize}
    \end{solution}
    

\end{homeworkProblem}