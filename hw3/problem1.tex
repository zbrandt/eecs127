\begin{homeworkProblem}

    Consider an  aerial system that moves in $\mathbb R^3$ according to the 
    dynamics
    \begin{equation}
        x(k+1)=Ax(k)+Bu(k), \quad k=0,1,2,3
    \end{equation}

    where $x(k)\in\mathbb R^3$ is the position of the system at time $k \in 
    \{0, 1, 2, 3, 4\}$ and $u(k) \in \mathbb R$ is the scalar input applied to 
    the system at time $k$. Assume that the initial position $x(0)$ is equal 
    to $[0 \ \ 0 \ \ 0]^T$. Given a target position $x_d\in\mathbb R^3$, the 
    goal is to design the input sequence $u(0), u(1),u(2),u(3)$ to take the 
    system to the target position $x_d$ at time $k=4$, i.e., $x(4)=x_d$. 

    \begin{itemize}
        \item [i)] Find a matrix $H \in \mathbb R^{3 \times 4}$ in terms of 
        $A$ and $B$ with the property that
        \begin{equation}
            x(4) = H \left[ \begin{array}{c} 
                        u(0) \\ 
                        u(1) \\ 
                        u(2) \\ 
                        u(3) \end{array} 
                    \right]
        \end{equation}
        
        \item [ii)] Assume that 
            \begin{equation}
                \label{eq1}
                A = \left[ \begin{array}{ccc} 
                        2 & 0 & 0 \\ 
                        -1 & 1 & 0 \\ 
                        -1 & -1 & 1 \end{array}
                    \right], \qquad 
                B = \left[ \begin{array}{c} 
                        1 \\ 
                        -1 \\ 
                        1 \end{array}
                    \right]
            \end{equation}
            
            Show that the vector $[ 1 \ \ 1\ \ 0]^T$ belongs to 
            $\mathcal N(H^T)$ (note: you are allowed to use a calculator to 
            compute $H$, but you cannot use a calculator or a computer code 
            to study the null space of $H^T$ and the analysis should be done 
            by hand). 

        \item [iii)] Again, consider the system parameters given in
            ~\eqref{eq1}. By studying the relationship between 
            $\mathcal N(H^T)$ and $ \mathcal R(H)$, prove that there is no 
            sequence of inputs that can take the system to the position 
            $x_d=[ 1 \ \ 1\ \ 0]^T$ at time $4$.

        \item [iv)] Again, consider the system parameters given in
            ~\eqref{eq1}. By finding $\mathcal N(H^T)$ and using the relations 
            $\mathcal N(H^T) \perp \mathcal R(H)$ and $\mathcal N(H^T) \oplus 
            \mathcal R(H)= \mathbb R^3$, show that there exists a sequence of 
            inputs to  take the system to the position $x_d$ at time $4$ if 
            and only if $x_d$ belongs to the set

            \begin{equation}
                \{x\in\mathbb R^3\ | \ x_1+x_2=0\}
            \end{equation}

        \item [v)] \textbf{(Coding)} Now, assume that 
            \begin{equation}
                A= \left[ \begin{array}{ccc} 
                    2 & 0 & 0 \\ 
                    -1 & 1 & 0 \\ 
                    -1 & -1 & 1 \end{array}
                    \right], \qquad 
                B= \left[ \begin{array}{c} 
                    -1 \\ 
                    -1 \\ 
                    1 \end{array}
                    \right]
            \end{equation}

            The goal is to find a sequence of inputs such that the total energy 
            $u(0)^2 + u(1)^2 + u(2)^2 + u(3)^2$ is minimized and yet the system 
            arrives at the target position $x_d=[ 3 \ \ 2\ \ 2]^T$ at time $4$. 
            Formulate this as an optimization problem and write a code in CVX 
            to solve the problem numerically. Plot the optimal trajectory 
            (i.e., plot the optimal values of the points $x(0),...,x(4)$ in 
            $\mathbb R^3$ and then connect each point to the next one (such 
            as $x(1)$ to $x(2)$)).
        
        \item [vi)] \textbf{(Coding)} Consider the safety set 
            \begin{equation}
                \mathcal S=\{x \in \mathbb R^3 \ | \ -3.3 \leq x_i \leq 3.2,
                \quad i=1,2,3\}
            \end{equation}
            
            Assume that the state $x(k)$ must always stay in the safety set 
            $\mathcal S$ for $k=0,1,...,4$. Redo Part (v) under this additional 
            constraint and find the optimal input sequence. Compares the 
            optimal trajectories and optimal energies (objective values) 
            obtained in Parts (v) and (vi). 
    \end{itemize}


\end{homeworkProblem}