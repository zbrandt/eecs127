\begin{homeworkProblem}

    Answer the following questions about SVD (note: all calculations 
    should be done by hand):
    \begin{itemize} 
        \item [i)] Consider the matrix
        \begin{equation}
            A=\left[\begin{array}{ccc} 2 & -1 & 2\\ -1 & 2 & 2\\ 
            2 & 2 & -1\end{array}\right]
        \end{equation}
        Show that the columns of $A$ are orthogonal to each other. By using 
        this fact, find a singular value decomposition of $A$.

        \item [ii)] Find a singular value decomposition of the matrix 
        \begin{equation}
            C= \left[\begin{array}{ccc} 0 & 0 & 1\\ 0 & 2 & 0\\3 & 0& 0
            \end{array}\right]\times \left[\begin{array}{ccc} 2 & -1 & 2\\ 
            -1 & 2 & 2\\ 2 & 2 & -1\end{array}\right]
        \end{equation}

        \item [iii)] Consider the optimization problem
        \begin{equation}
            \min_{B\in\mathbb R^{3\times3}} \|C-B\|_F\qquad\text{s.t.}\qquad 
            \text{rank}(B)\leq 2
        \end{equation}
        Find the optimal solution $B^*$ and compute the error 
        $\frac{||C-B^*\|_F^2}{\|C\|_F^2}$.

    \end{itemize}

    \begin{solution}
        \begin{itemize}
            \item[i)] If the dot product of each pair of columns in $A$ is 
                zero, then the columns are orthogonal.
                \[
                    a_1 = \begin{bmatrix} 2 \\ -1 \\ 2 \end{bmatrix}, \quad
                    a_2 = \begin{bmatrix} -1 \\ 2 \\ 2 \end{bmatrix}, \quad
                    a_3 = \begin{bmatrix} 2 \\ 2 \\ -1 \end{bmatrix}
                \]
                \[
                    \begin{split}
                        a_1 \cdot a_2 &= 2 \cdot (-1) + (-1) \cdot 2 + 2 \cdot 2 = -2 - 2 + 4 = 0 \\
                        a_1 \cdot a_3 &= 2 \cdot 2 + (-1) \cdot 2 + 2 \cdot (-1) = 4 - 2 - 2 = 0 \\
                        a_2 \cdot a_3 &= (-1) \cdot 2 + 2 \cdot 2 + 2 \cdot (-1) = -2 + 4 - 2 = 0
                    \end{split}
                \]
                
                Since the dot products are all equal to zero, the columns of
                $A$ are orthogonal. Using this fact, the singular values of $A$
                are the norms of its columns, since $A^\top A$ is a diagonal 
                matrix with the squared norms of the columns on the diagonal.
                \[
                    \begin{split}
                        \sigma_1 &= \|a_1\| = \sqrt{2^2 + (-1)^2 + 2^2} = \sqrt{9} = 3 \\
                        \sigma_2 &= \|a_2\| = \sqrt{(-1)^2 + 2^2 + 2^2} = \sqrt{9} = 3 \\
                        \sigma_3 &= \|a_3\| = \sqrt{2^2 + 2^2 + (-1)^2} = \sqrt{9} = 3
                    \end{split}
                \]

                The singular value decomposition of $A$ is given by 
                $A = \mathbf{U} \tilde{\mathbf{\Sigma}} \mathbf{V}^\top$, where
                $\tilde{\mathbf{\Sigma}}$ is a diagonal matrix with the singular
                values on the diagonal, and $\mathbf{U}$ and $\mathbf{V}$ are 
                orthogonal matrices. The columns of $\mathbf{V}$ are the 
                normalized eigenvectors of $A^\top A$. But, since $A^\top A$ is
                a scalar multiple of the identity matrix, any vector in
                $\mathbb{R}^3$ is an eigenvector, $A^\top A v_i = 9 v_i$. 
                Therefore, the standard basis vectors can be used as the 
                columns of $\mathbf{V}$:
                \[
                    \mathbf{V} = \begin{bmatrix}
                        1 & 0 & 0 \\
                        0 & 1 & 0 \\
                        0 & 0 & 1
                    \end{bmatrix}
                \]

                The matrix $\tilde{\mathbf{\Sigma}}$ is a diagonal matrix with the
                singular values on the diagonal:
                \[
                    \tilde{\mathbf{\Sigma}} = \begin{bmatrix}
                        3 & 0 & 0 \\
                        0 & 3 & 0 \\
                        0 & 0 & 3
                    \end{bmatrix}
                \]

                The matrix $\mathbf{U}$ is obtained by normalizing the columns of $A$
                since $u_i = \frac{Av_i}{\sqrt{\lambda_i}} = 
                \frac{a_i}{\sqrt{\|a_i\|}}$:
                \[
                    \mathbf{U} = \begin{bmatrix}
                        \frac{2}{3} & -\frac{1}{3} & \frac{2}{3} \\
                        -\frac{1}{3} & \frac{2}{3} & \frac{2}{3} \\
                        \frac{2}{3} & \frac{2}{3} & -\frac{1}{3}
                    \end{bmatrix}
                \]

                Therefore, the singular value decomposition of $A$ is:
                \[
                    A = \begin{bmatrix}
                        \frac{2}{3} & -\frac{1}{3} & \frac{2}{3} \\
                        -\frac{1}{3} & \frac{2}{3} & \frac{2}{3} \\
                        \frac{2}{3} & \frac{2}{3} & -\frac{1}{3}
                    \end{bmatrix}
                    \begin{bmatrix}
                        3 & 0 & 0 \\
                        0 & 3 & 0 \\
                        0 & 0 & 3
                    \end{bmatrix}
                    \begin{bmatrix}
                        1 & 0 & 0 \\
                        0 & 1 & 0 \\
                        0 & 0 & 1
                    \end{bmatrix}^\top
                \]

            \item[ii)] The matrix $C$ is given by the product of two
                matrices: 
                \[
                    C = \begin{bmatrix}
                        0 & 0 & 1 \\
                        0 & 2 & 0 \\
                        3 & 0 & 0
                    \end{bmatrix}
                    \begin{bmatrix}
                        2 & -1 & 2 \\
                        -1 & 2 & 2 \\
                        2 & 2 & -1
                    \end{bmatrix}
                    = \begin{bmatrix}
                        2 & 2 & -1 \\
                        -2 & 4 & 4 \\
                        6 & -3 & 6
                    \end{bmatrix}
                \]

                The matrix $C^\top C$ is then:
                \[
                    C^\top C = \begin{bmatrix}
                        2 & -2 & 6 \\
                        2 & 4 & -3 \\
                        -1 & 4 & 6
                    \end{bmatrix}
                    \begin{bmatrix}
                        2 & 2 & -1 \\
                        -2 & 4 & 4 \\
                        6 & -3 & 6
                    \end{bmatrix}
                    = \begin{bmatrix}
                        44 & -22 & 26 \\
                        -22 & 29 & -4 \\
                        26 & -4 & 53
                    \end{bmatrix}
                \]

                The eigenvalues of $C^\top C$ are found by solving the 
                characteristic polynomial $\det(C^\top C - \lambda I) = 
                0$. The eigenvalues are $\lambda_1 = 81$, 
                $\lambda_2 = 36$, and $\lambda_3 = 9$. The singular values 
                are the square roots of the eigenvalues:
                \[
                    \sigma_1 = \sqrt{81} = 9, \quad
                    \sigma_2 = \sqrt{36} = 6, \quad
                    \sigma_3 = \sqrt{9} = 3
                \]

                The matrix $\tilde{\mathbf{\Sigma}}$ is then:
                \[
                    \tilde{\mathbf{\Sigma}} = \begin{bmatrix}
                        9 & 0 & 0 \\
                        0 & 6 & 0 \\
                        0 & 0 & 3
                    \end{bmatrix}
                \]

                The columns of $\mathbf{V}$ are the normalized eigenvectors of 
                $C^\top C$, and the columns of $\mathbf{U}$ are given by $u_i = 
                \frac{Cv_i}{\sigma_i}$. The eigenvectors corresponding to 
                $\lambda_1$, $\lambda_2$, and $\lambda_3$ are then the 
                following respectively:
                \[
                    e_1 = \begin{bmatrix} 1 \\ -\frac{1}{2} \\ 1 \end{bmatrix}, \quad
                    e_2 = \begin{bmatrix} -\frac{1}{2} \\ 1 \\ 1 \end{bmatrix}, \quad
                    e_3 = \begin{bmatrix} -2 \\ -2 \\ 1 \end{bmatrix}
                \]

                The matrix $\mathbf{V}$ is then formed by normalizing these 
                eigenvectors:
                \[
                    \mathbf{V} = \begin{bmatrix}
                        \frac{2}{3} & -\frac{1}{3} & -\frac{2}{3} \\
                        -\frac{1}{3} & \frac{2}{3} & -\frac{2}{3} \\
                        \frac{2}{3} & \frac{2}{3} & \frac{1}{3}
                    \end{bmatrix}
                \]

                The matrix $\mathbf{U}$ is then formed by computing $u_i = 
                \frac{Cv_i}{\sigma_i}$ for each $i$:
                \[
                    \mathbf{U} = \begin{bmatrix}
                        0 & 0 & -1 \\
                        0 & 1 & 0 \\
                        1 & 0 & 0
                    \end{bmatrix}
                \]

                Therefore, the singular value decomposition of $C$ is:
                \[
                    C = \begin{bmatrix}
                        0 & 0 & -1 \\
                        0 & 1 & 0 \\
                        1 & 0 & 0
                    \end{bmatrix}
                    \begin{bmatrix}
                        9 & 0 & 0 \\
                        0 & 6 & 0 \\
                        0 & 0 & 3
                    \end{bmatrix}
                    \begin{bmatrix}
                        \frac{2}{3} & -\frac{1}{3} & -\frac{2}{3} \\
                        -\frac{1}{3} & \frac{2}{3} & -\frac{2}{3} \\
                        \frac{2}{3} & \frac{2}{3} & \frac{1}{3}
                    \end{bmatrix}^\top
                \]

                \item[iii)] The optimal solution $B^*$ to the optimization
                    problem is given by the truncated singular value 
                    decomposition of $C$, keeping only the top two singular 
                    values and the first two columns of $\mathbf{U}$ and 
                    $\mathbf{V}$:
                    \[
                        B^* = \mathbf{U_2} \mathbf{\tilde{\Sigma}_2} \mathbf{V_2}^\top
                    \]
                    where $\mathbf{U_2}$, $\mathbf{\tilde{\Sigma}_2}$, and 
                    $\mathbf{V_2}$ are the matrices formed by taking the first 
                    2 columns of $\mathbf{U}$, the first 2 singular values in
                    $\mathbf{\tilde{\Sigma}}$, and the first 2 columns of 
                    $\mathbf{V}$, respectively.
                    \[
                        \mathbf{U_2} = \begin{bmatrix}
                            0 & 0 \\
                            0 & 1 \\
                            1 & 0
                        \end{bmatrix}, \quad
                        \mathbf{\tilde{\Sigma}_2} = \begin{bmatrix}
                            9 & 0 \\
                            0 & 6
                        \end{bmatrix}, \quad
                        \mathbf{V_2} = \begin{bmatrix}
                            \frac{2}{3} & -\frac{1}{3} \\
                            -\frac{1}{3} & \frac{2}{3} \\
                            \frac{2}{3} & \frac{2}{3}
                        \end{bmatrix}
                    \]

                    The optimal solution $B^*$ is then:
                    \[
                        B^* = \begin{bmatrix}
                            0 & 0 \\
                            0 & 1 \\
                            1 & 0
                        \end{bmatrix}
                        \begin{bmatrix}
                            9 & 0 \\
                            0 & 6
                        \end{bmatrix}
                        \begin{bmatrix}
                            \frac{2}{3} & -\frac{1}{3} \\
                            -\frac{1}{3} & \frac{2}{3} \\
                            \frac{2}{3} & \frac{2}{3}
                        \end{bmatrix}^\top
                    \]
                    \[
                        = \begin{bmatrix}
                            0 & 0 & 0 \\
                            -2 & 4 & 4 \\
                            6 & -3 & 6
                        \end{bmatrix}
                    \]

                    The squared Frobenius norm of the difference $C - B^*$ is
                    the sum of remaining squared singular value $3^2 = 9$. The 
                    squared Frobenius norm of $C$ is the sum of squared
                    singular values:
                    \[
                        \|C\|_F^2 = \sigma_1^2 + \sigma_2^2 + \sigma_3^2
                            = 9^2 + 6^2 + 3^2 = 81 + 36 + 9 = 126
                    \]
                    Therefore, the error is:
                    \[
                        \frac{\|C - B^*\|_F^2}{\|C\|_F^2} = \frac{9}{126} = \frac{1}{14}
                    \]
            \end{itemize}
    \end{solution}

\end{homeworkProblem}