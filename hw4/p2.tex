\begin{homeworkProblem}

    An exam with $m$ questions is given to $n$ students. The instructor 
    collects all the grades in an $n\times m$ matrix $G$ where $G_{ij}$ 
    shows the grade of student $i\in\{1,2,...,n\}$ for question 
    $j\in\{1,2....,m\}$. By analyzing the matrix $G$, the goal is to 
    design a difficulty score for each question that shows the difficulty 
    level of that question. As a naive approach, one may consider the 
    average grade $\frac{\sum_{i=1}^n G_{ij}}{n}$ as the difficulty score 
    of question $j$. To understand the issue with this difficulty score, 
    assume that $n=m=2$ and $G$ is equal to
    \begin{equation}
        G=\left[\begin{array}{cc} 50& 100 \\ 50 & 0 \end{array}\right]
    \end{equation}
    where the minimum and maximum grades for each question are 0 and 100. 
    In this example, both questions have the same average grade of 50. Both 
    students have done poorly on question 1. For question 2, one student 
    got the highest grade possible while the other student got 0 (which may 
    imply that the student was not prepared for that question rather than 
    the question being hard). Question 1 seems to be much harder than 
    question 2 due to the distribution of the grades while the average 
    grades cannot provide any useful information. To address this issue for 
    arbitrary values of $n$ and $m$, we propose an optimization model for 
    the design of difficulty scores.
    \begin{itemize}
        \item [i)] Consider the optimization problem
        \begin{equation}
            \min_{B\in\mathbb R^{n\times m}} \|G-B\|_F\qquad\text{s.t.}\qquad 
            \text{rank}(B)\leq 1
        \end{equation}
        We decompose the optimal solution $B^*$ as $xy^T$, where 
        $x \in \mathbb R^n$ and $y \in \mathbb R^m$. Assume that $x,y\geq 0$ 
        (note: if no student receives a zero score on any question, then it can 
        be proven that $x$ and $y$ are automatically nonnegative vectors). 
        Assume that the error $\frac{||G-B^*\|_F^2}{\|G\|_F^2}$ is small. 
        Explain how a difficulty score can be designed for each question in 
        terms of $x$ and $y$.

        \item [ii)] Consider the case with $n=3$ and $m=5$, together with the 
        grade matrix
        \begin{equation}
            G=\left[\begin{array}{ccccc} 100 & 90 & 100 & 80 & 70 \\ 
            80 & 70& 60& 70& 80\\ 60& 50& 40& 50 & 60 \end{array}\right]
        \end{equation}
        Using Part (i), design a difficulty score for each question, and rank 
        the questions from the hardest to the easiest based on their scores 
        (note: you can use a computer code for SVD calculations). 

    \end{itemize}

    \begin{solution}
        \begin{itemize}
            \item[i)] The optimal solution $B^* = xy^\top$ is attained for $x =
                \sqrt{\sigma_1} u_1$ and $y = \sqrt{\sigma_1} v_1$, where 
                $\sigma_1$ is the largest singular value of $G$, and $u_1$ and 
                $v_1$ are the corresponding left and right singular vectors, 
                respectively. Since the error is small, each predicted grade 
                for student $i$ on question $j$ is given by $B^*_{ij} = 
                x_i y_j$. Since $x \in \mathbb{R}^n$, each entry $x_i$ can be 
                interpreted as a measure of the ability of student $i$, for a
                constant $y_j$. Similarly, since $y \in \mathbb{R}^m$, each
                entry $y_j$ can be interpreted as a measure of the difficulty
                of question $j$, for a constant $x_i$. Therefore, a difficulty
                score for each question could be of the form $z_j = 1/y_j$,
                where a higher value of $z_j$ indicates a more difficult
                question.

            \item[ii)] The compact, rank-1 singular value decomposition of $G$
                is given by
                \[
                    G \approx B^* = \sigma_1 u_1 v_1^\top = 280.216
                    \begin{bmatrix}
                        -0.704 \\ -0.575 \\ -0.416
                    \end{bmatrix}
                    \begin{bmatrix}
                        -0.505 & -0.444 & -0.434 & -0.419 & -0.429
                    \end{bmatrix}
                \]

                where $G^\top G$ is given by
                \[
                    G^\top G = \begin{bmatrix}
                        20000 & 17600 & 17200 & 16600 & 17000 \\
                        17600 & 15500 & 15200 & 14600 & 14900 \\
                        17200 & 15000 & 15200 & 14200 & 14200 \\
                        16600 & 14200 & 14200 & 13800 & 14200 \\
                        17000 & 14900 & 14200 & 14200 & 14900
                    \end{bmatrix}
                \]

                and has an eigenvalue of $\lambda_1 = 78521.051$ with 
                corresponding normalized eigenvector
                \[
                    v_1 = \begin{bmatrix}
                        -0.505 \\ -0.444 \\ -0.434 \\ -0.419 \\ -0.429
                    \end{bmatrix}
                \]

                The largest singular value of $G$ is given by $\sigma_1 =
                \sqrt{\lambda_1} = 280.216$. The corresponding left singular
                vector is given by $u_1 = \frac{1}{\sigma_1} G v_1 \approx
                \begin{bmatrix} -0.420 \\ -0.336 \\ -0.252 \end{bmatrix}$

                Therefore, the difficulty score for each question is given by
                \[
                    z = \begin{bmatrix}
                        1/(\sqrt{\sigma_1} v_1)_1 \\
                        1/(\sqrt{\sigma_1} v_1)_2 \\
                        1/(\sqrt{\sigma_1} v_1)_3 \\
                        1/(\sqrt{\sigma_1} v_1)_4 \\
                        1/(\sqrt{\sigma_1} v_1)_5
                    \end{bmatrix} =
                    \begin{bmatrix}
                        1/(-4.749) \\ 1/(-4.176) \\ 1/(-4.084) \\ 1/(-3.942) \\ 1/(-4.037)
                    \end{bmatrix} \approx
                    \begin{bmatrix}
                        -0.210 \\ -0.240 \\ -0.245 \\ -0.254 \\ -0.248
                    \end{bmatrix}
                \]
                The questions ranked from hardest to easiest are 1, 2, 5, 3, 4.
        \end{itemize}

    \end{solution}

\end{homeworkProblem}